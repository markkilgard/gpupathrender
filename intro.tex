
\section{Introduction}
\label{sec:intro}

%% The ``\copyrightspace'' command must be the first command after the 
%% start of the first section of the body of your paper. It ensures the
%% copyright space is left at the bottom of the first column on the first
%% page of your paper.

\copyrightspace

When people surf the web, read PDF documents, interact with smart
phones or tablets, use productivity software, play casual games, or
in everyday life encounter the full variety of visual output created
digitally (advertising, books, logos, signage, etc.), they are experiencing
resolution-independent 2D graphics.

While 3D graphics dominates graphics research, we observe that most visual
interactions between humans and computers involve 2D graphics.  Sometimes
this type of computer graphics is called {\em vector graphics}, but we
prefer the term {\em path rendering} because the latter term emphasizes
the path as the unifying primitive for this approach to rendering.

\subsection{Terminology of Path Rendering}

A {\em path} is a sequence of trajectories and contours.  In this context,
a {\em trajectory} is a connected sequence of {\em path commands}.
Path commands include line segments, B\'{e}zier curve segments, and
partial elliptical arcs.  Each path command has an associated set of
numeric parameters known as {\em path coordinates}.  When a pair of path
coordinates defines a 2D $(x,y)$ location, this pair is a {\em control
point}.  Intuitively a trajectory corresponds to pressing a pen's tip
down on paper, dragging it to draw on the paper, and eventually lifting
the pen.

A {\em contour} is a trajectory with the same start and end point; in other
words, a closed trajectory.  These contours and trajectories may be
convex, self-intersecting, nested in other contours, or may intersect
other trajectories/contours in the path.  There is generally no bound
on the number of path segments or trajectories/contours in a path.
For a rendering ``primitive,'' paths can be quite complex.

Paths are rendered by either {\em filling} or {\em stroking} the path.
Conceptually, path filling corresponds to determining what points
(framebuffer sample locations) are logically ``inside'' the path.
Stroking is roughly the region swept out by a fixed-width pen---centered
on the trajectory---that travels along the trajectory orthogonal to the
trajectory's tangent direction.

\subsection{History, Standards, Motivation, and Contributions}

Seminal work by Warnock and Wyatt
\shortcite{Warnock:1982:DIG:800064.801297} introduced a coherent model
for path rendering.  Since that time, many standards and programming
interfaces have incorporated path rendering constructs into their
2D graphics framework.  Without being exhaustive, we note
%\vspace{-4pt}
\begin{itemize}
\item {\em document presentation and printing}: PostScript \cite{PLRM}, PDF \cite{PDF-Spec}
\item {\em font specification}: PostScript fonts \cite{Type1-Spec}
\item {\em immersive web}: Flash \cite{SWF-File-Format}, HTML 5's Scalable Vector Graphics \cite{SVG-Spec}
\item {\em 2D programming interfaces}: OpenVG \cite{OpenVG-Spec}
\item {\em productivity software}: Illustrator, Photoshop, Office
\end{itemize}
%\vspace{-1pt}
Despite path rendering's 30 year heritage and broad adoption, it has
not benefited from acceleration by graphics hardware to anywhere near 
the extent 3D graphics has.  Most path rendering today is performed
by the CPU with sequential algorithms, not particularly different from
their formulation 30 years ago.  Our motivation is to harness {\em existing}
GPUs to improve the overall experience achievable with path rendering.

We present a productized system for GPU-accelerated path rendering in
the context of the OpenGL graphics system; see some of our rendering results
in Figure~\ref{fig:teaser}.  Our system works on the three most-recent
architectural generations of GeForce and Quadro GPUs---and we expect
all recent GPUs can support the algorithms and programming interface
we describe.

The primary contributions delivered by our system are:
%\vspace{-4pt}
\begin{itemize}
  \item A novel "stencil, then cover" programming interface
  for path rendering, well-suited to acceleration by GPUs.
  \item Our programming interface's efficient implementation within the
  OpenGL graphics system to avoid CPU bottlenecks.
  \item Accompanying algorithms to handle tessellation-free stenciled stroking 
  of paths, standard stroking embellishments such
  as dashing, clipping paths to arbitrary paths, and mixing 3D and path
  rendering.
\end{itemize}
%\vspace{-1pt}
Section~\ref{sec:prior} reviews prior path rendering systems.
Section~\ref{sec:approach} explains our approach;
we cite the crucial prior research that our approach integrates in this section.
Section~\ref{sec:discuss} compares our quality and performance to other
implementations and highlights our system's novel ability to mix with 3D
and GPU-shaded rendering.  Section~\ref{sec:future} discusses opportunities for
future work.

\subsection{New Demands on Path Rendering}

Historically, applications mostly ``pre-render'' 2D content specified
with paths into bitmaps for glyphs and icons/images for vector artwork,
then cache and blit those rasterized results as needed.  Rendering
directly from the path data generally proved too slow to be viable.
Early window systems based on path rendering concepts such as Sun's NeWS
\cite{NeWS-Book} and Adobe's Display PostScript \cite{DisplayPostScript}
were arguably overly ambitious in basing their 2D rendering model around
path rendering rather than resolution-dependent 2D bitmap rendering as
did the more successful GDI and X11-based systems that proved easier
for 2D graphics hardware to accelerate.

\subsubsection{Increasing Screen Density and Resolution}

Smart phones and tablets have created new platforms
free from legacy display limitations such as relatively low---by today's available technology---display density
(measured in {\em dots-per-inch} or DPI) and the dated visual appearance established
by resolution-dependent bitmap graphics.  
Apple's new iPad display has a display density of 264 DPI, greatly surpassing
the 100 DPI density norm for PC screens.  These handheld devices
are carried directly on one's person so their screen real estate is
relatively fixed---so improvements in display appearance is likely to
be through increasing screen density rather than enlarging screen area.

Pixel resolutions for conventional monitors are increasing too.
Large 2560x1600 resolution screens are mass-produced and readily
available.  Driving such high resolutions with CPU-based path rendering
schemes is untenable at real-time rates.  Indeed the very heterogeneity of modern displays in terms of
pixel resolution, screen size, and their combination---pixel
density---strengthens the case for improving path rendering performance.

\subsubsection{Multi-touch Interfaces}

Mobile devices also rely on multi-touch screens for input so the user is
extremely aware of the latency between touch gestures and the resulting
screen update.  The user is literally pointing at the pixels they expect
to see updated.  Multi-touch encourages rotation and scaling.
When imagery can easily be rotated, scaled, sub-pixel translated, and even
projected, assumptions that all text and graphics will be orthographically aligned to
the screen's pixel grid are no longer a given so
rendering all path content directly from paths makes sense.

\subsubsection{Immersive Web Standards}

The proximate HTML 5 web standard exposes path rendering functionality
in a standard and pervasive way through both Scalable Vector Graphics (SVG)
and the Canvas element.

JavaScript performance has increased to the point that dynamic content
can be orchestrated within a standards-based HTML 5 web page such that
the system's path rendering performance is often a bottleneck.

\subsubsection{Power Wall}

Minimizing power consumption has become a mantra for computer system
design across all classes of devices---whether mobile devices or not.
When power is at a premium, moving CPU- and bandwidth-intensive
computations such as pixel manipulation and rasterization to
more power-efficient GPU circuitry can reduce overall power
consumption while improving interactivity and minimizing update latency.
GPU-acceleration of path rendering is precisely such an opportunity.

