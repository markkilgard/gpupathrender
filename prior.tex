
\section{Prior Path Rendering Systems}
\label{sec:prior}

\subsection{CPU-based Path Rendering Systems Critiqued}

Path rendering is historically and still typically performed by
CPU-based scan line renderers.  Paths are converted into a set of edges.
These edges are transformed and sorted in Y-order.  Then the edges are
intersected with each scan line, sorted in X-order, and pixels in the
path region are updated.

The scan-line rendering approach is notable for being work-efficient
and cache-friendly.  No computation is expended on pixels that are
obviously outside the path and only active edges are considered when
processing a given scan line.  Such scan line renderers use a ``chunker''
strategy---where rather than the chunk being a 2D tile, the chunk is a
single scan line.  This leads to a reasonably friendly access pattern for
CPU caches.  Additionally the scan line enter/leave counts are transient.
In contrast to a window-sized ancillary buffer such as a depth or stencil
buffer, the scan line enter/leave counts can live in the cache and have
their storage recycled for each processed scan line.

While work-efficient {\em and} cache-friendly as noted, this CPU-intensive
approach is quite sequential.  Every path must be transformed into screen
space.  Every path must be scan line rasterized.  Every scan line must
be intersected with the active edge list.  Every sorted active edge list
must be scanned left-to-right.  There is not an easy way to pipeline
all these tasks or exploit massive parallelism---such as is routine for
GPU-accelerated 3D graphics.  Hence this is an approach that maps well
to the CPU but cannot be obviously accelerated in this form by the GPU.

\subsection{GPU-based Path Rendering Systems}

Over the years, many attempts have been made---with varying degrees of
mixed success---to accelerate path rendering with GPUs.
We postpone discussion of prior techniques for GPU rendering of curves with the stencil buffer
to Section~\ref{sec:approach} since they are the basis for our approach.

\subsubsection{Acceleration of Path Rendering Programming Interfaces}

Cairo \cite{Rejected} is an open-source path rendering implementation.
An early attempt at GPU-acceleration called Glitz \cite{Glitz} has
since been abandoned.  Glitz operated at the level of the XRender
\cite{Packard:2001:DIX:647054.715760} extension so did not accelerate
paths directly.  Arguably, Glitz was a more GPU-assisted back-end than
GPU accelerated. 
More recently, Cairo has worked on a first-class GPU back-end but the
immediate mode nature of the Cairo API and converting CPU-transformed
paths to spans limits the acceleration opportunities.  

Microsoft's Direct2D \cite{Direct2D} API is layered upon Direct3D.
Direct2D operates by transforming paths on the CPU and then performing
a constrained trapezoidal tessellation of each path.  The result is a
set of pixel-space trapezoids and additional shaded geometry to compute
fractional coverage for the left and right edges of the trapezoids.
These trapezoids and shaded geometry are then rasterized by the GPU.
The resulting performance is generally better than entirely CPU-based
approaches and requires no ancillary storage for multisample or stencil
state; Direct2D renders directly into an aliased framebuffer with
properly antialiased results.  Direct2D's primary disadvantage is the
ultimate performance is determined not by the GPU (doing fairly trivial
rasterization) but rather by the CPU performing the transformation and
trapezoidal tessellation of each path and Direct3D validation work.

Skia is the C++ path rendering API used by Google's Android and Chrome
browsers.  Skia has a conventional CPU-based path renderer but has
recently integrated a new OpenGL ES2-accelerated back-end called Ganesh.
Ganesh has experimented with two accelerated approaches.  The first
used the stencil buffer to render paths.  Because of API overheads
with this approach, this first approach was replaced with a second
approach where the CPU-based rasterizer computes a coverage mask which
is loaded as a texture upon every path draw to provide the GPU proper
antialiased coverage.  This hybrid scheme is often bottlenecked by the
dynamic texture updates required for every rendered path.

The Khronos standards body worked to develop an API standard known
as OpenVG with the explicit goal of enabling hardware-acceleration of
path rendering (the {\em VG} stands for vector graphics).
Various companies and research groups have worked to develop OpenVG
hardware designs \cite{iMX35,Huang:2006:IOR:1169227.1169771,4637621} that,
based on available descriptions, are fairly similar to the conventional
CPU-based scan line rasterizer scheme, recast as a hardware unit.
Reported performance levels are quite low compared to what we report.

\subsubsection{Vector Texture Schemes}

An unconventional approach to GPU-accelerating path rendering
is cleverly encoding path content into GPU memory---typically as a
texture---and then using a programmable shader essentially to decode the
path content.  Nehad and Hoppe \shortcite{Nehab:2008:RRG:1457515.1409088}
and Qin \shortcite{QinThesis} adopt variations on this approach.  While this
approach has some interesting advantages such as being able to directly
``texture map'' 3D geometry with path rendered content, these approaches
suffer from the need to preprocess a static path scene into a specific
texture encoding.  This makes this approach unsuitable for editable or
dynamic path rendering.  Additionally, many rendering approximations
and authoring limitations are needed to make vector texture schemes
tractable.

\subsubsection{Discussion of Deficiencies}

The norm for CPU-based path rendering systems is maintaining roughly
16 coverage samples per pixel (details vary).  This creates a challenge
for GPU-based schemes because GPUs often support 1, 2, 4, or 8 samples
per pixel through multisampling.  This often creates a situation where
the GPU-accelerated path rendering is inferior to the CPU-based path
rendering quality.

When path rendering schemes are layered upon existing OpenGL or Direct3D
APIs, we have observed performance being limited by the state change
rate of the underlying 3D API.  Often path rendering can result in
many state changes per path when scenes can easily consist of 100s or
1000s of paths.  In this case, the API overhead can substantially limit
the overall performance.  Our experience studying prior approaches to
using GPUs for path rendering indicates these approaches are often more
GPU-assisted rather than GPU-accelerated, with this attributable to
continuous CPU involvement or substantial CPU-based preprocessing.

